
\section{Editor: how to?}
\index{secrets!Editor: how to?}

Here we describe some tips.

\subsection{Macros: how to use?}
\index{secrets!macros}

Tinn-R provides a useful macro option to help you with repetitive actions.\\

\vspace{5mm}
\includegraphics[scale=1]{./res/secrets_macro.png}\\
\vspace{5mm}

Click on \texttt{Tools/Macro} at the main menu bar. Then you will see see two icons: Record \texttt{F7} and Play \texttt{F8}.
Those icons may also be found on the editor task bar. Here is an example: write a vector 1:20 and send it to \RR{}.
In the \texttt{Term} window you will see the numbers from 1 to 20 in a single line. Copy and paste that to the editor's window.

Suppose that you want to write a comma after every number, maintaining a one space distance after the comma.
Place the cursor just before the number one and then press \texttt{F7}.
A \texttt{+} sign will appear on the Record icon at the editor taskbar, meaning that you are beginning to record action
you take on the screen. Press the right arrow to move across the number 1 and type a comma (,),
delete one space and move the cursor one space to the right with the right arrow and press again
\texttt{F7} to stop the recording process. The little \texttt{+} sign will disappear.
Now just press \texttt{F8} and see what happens. When you reach the number ten it won't work because now the numbers
consists of two digits and the spacings will not fit.
After doing that for the number nine the cursor will stop in the middle of the number ten,
go back one step to the left with the left arrow, and press space bar so that there will be one blank space
between the comma and the number ten. Now press again \texttt{F7} to go across the number ten with the right arrow,
type the comma (,) and go one space to the right with the right arrow so the cursor is now at the left side of
the number eleven, press again \texttt{F7}, and then \texttt{F8} until the end.
Try now to write those numbers in a vertical line across the left margin without the commas using the macros again .

\subsection{Marks: how to use?}
\index{secrets!marks}

One very useful navegation tool is the \texttt{bookmark}. To define the bookmark,
use \texttt{CTRL + SHIFT + [0-9]} (a key from 0 to 9) on the line you have chosen.
Then, to go to the corresponding bookmark just use \texttt{CTRL + [0-9]}.
A visual indicator appears in the left margin, just before the line number, of that line.
Suppose you are working on a long script and need to constantly return to a specified line.
You will see how handy it is. To undo it, just use again \texttt{CTRL + SHIFT + [0-9]} using the same number, on the same line.

You may also mark a whole block. To to that, first select the block and then click on Marks at the main menu and click on mark.
As in the above paragraph the number 0 will appear at the begining of the block and the number 1 at the end.
You can modify the box and send it to \RR{} to be processed.

\subsection{Editor, Tools and Term: how to arrange?}
\index{secrets!arrange}

\texttt{CTRL + F8} toggles the visibility of window \texttt{Tools} and \texttt{CTRL + F9}
toggles the visibility of window \texttt{Term}. Now press \texttt{CTRL + F10},
\texttt{CTRL + F11} or \texttt{CTRL + F12} and see what happens.

The use of two monitors is recommended in case of intensive use.
Placing \RR{} (Rgui.exe or Rterm.exe) on a second monitor makes working with data analysis
and development tasks very comfortable.

\subsection{Active page using the keyboard: how to change?}
\index{secrets!active page}

\begin{verbatim}
CTRL + TAB         : Change sequentially the active page to the right
                     (requires more than one)
CTRL + SHIFT + TAB : Change sequentially the active page to the left
                     (requires more than one)
\end{verbatim}

\subsection{Tab order: how to change?}
\index{secrets!tab order}

Drag and drop the tab to the left or right.\\

\vspace{5mm}
\includegraphics[scale=0.8]{./res/filetabs.png}\\
\vspace{5mm}

\subsection{Auto completion: how to use?}
\index{secrets!auto completion}

In principle it just opens and closes, with just one click, parentheses, square brackets, brackets and quotation marks, i.e.: \verb|( [ { ' "|.
You just click, say, the open parentheses and the editor automatically completes the closing parentheses and after that puts the cursor in the middle
of the two.

However, it is not only this. If already exists a previous text doing just that might be a bit laborious,
since you have to delete one of the two elements (the subsequent to the cursor), to move the cursor
to the end of the text and then to close the element (parentheses, square brackets, brackets or quotation marks).
There is an alternative for this: select the previous text and type the open element, i.e: \verb|( [ { ' "|. It is all!

One of the most important feature of this resource is the following: you just write the expression you want to complete, i.e., \verb|( [ { ' "|
then you select the whole expression and press the element key.
\textit{The two elements will automatically be placed at the beginning and the end of the selection and the entire selection will be preserved.}

\textbf{Important note}: This can be done in both, the editor as well as in the \texttt{Term} window.
\\
\\
A single and illustrative example:
\begin{itemize}
  \item A previous text: \\
    \includegraphics[scale=0.8]{./res/completion_howto_01.png}
  \item type \texttt{sum} and after to select the text: \\
    \includegraphics[scale=0.8]{./res/completion_howto_02.png}
  \item type the open parentheses: \\
    \includegraphics[scale=0.8]{./res/completion_howto_03.png}
\end{itemize}

A more complex and illustrative example:
\begin{itemize}
  \item A previous text: \\
    \includegraphics[scale=0.8]{./res/completion_howto_04.png}
  \item select it: \\
    \includegraphics[scale=0.8]{./res/completion_howto_05.png}
  \item type the quotation mark: \\
    \includegraphics[scale=0.8]{./res/completion_howto_06.png}
  \item type \texttt{cat}: \\
    \includegraphics[scale=0.8]{./res/completion_howto_07.png}
  \item select the text: \\
    \includegraphics[scale=0.8]{./res/completion_howto_08.png}
  \item type the open parentheses: \\
    \includegraphics[scale=0.8]{./res/completion_howto_09.png}
\end{itemize}
