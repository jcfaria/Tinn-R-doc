  \item The \texttt{R Send smart} option was finally enabled. It took a lot of work to do and we don't consider it finished. This option is still experimental
    and should not be seen as a substitute for \texttt{R send: line} and other safer features like block marking or manual selection.
    Should be used when the cursor is in the middle of some complex context (pipe, function or common usage) and for some
    reason \textbf{the user does not want to place the cursor in the first position of a block or the cursor is at the end of a block and want to send the block above}.
    Given the diversity of forms and programming habits that \texttt{R} allows, we believe that it is almost impossible to implement an
    infallible algorithm to identify the text block. With this feature we don't want to encourage bad habits to work with the R interpreter.
    Our intention is to try to make it easier to use in some common situations in order to increase a little efficiency in sending instructions.
    Due to the countless possibilities of error in the algoritmo, a opção padrão é apenas fazer a seleção do bloco, pois se errar o usuário pode corrigir
    manualmente a seleção. Para o envio da seleção, pode-se usar o recurso de enviar seleção ou o atalho default \texttt{CTRL+R}.
    Em \texttt{Tools/Misc/Project/Open demo}. Entretanto, em \texttt{Options/Application/R/Basic/Startup and running/After selecting} existem opções
    para customisar o recurso. Examples below:
  \verbatim
      res_c <- cut(res,
               .|  breaks = c(-Inf,
               .|             120,
               .|             130,
               .|             140,
    cursor pos .|             150,
               .|             Inf),
               .|  labels = c('black',
               .|             'red',
               .|             'green',
               .|             'blue',
               .|             'gold'))
               .|
  \end{verbatim}
