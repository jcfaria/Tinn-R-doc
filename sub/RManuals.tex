
\chapter{R manuals on CRAN}


The \RR{} core team creates several
\href{http://cran.r-project.org/manuals.html}{manuals}
for working with \RR{}\footnote{
  The manuals are created on Debian Linux and may differ from the manuals for Mac
  or Windows on platform-specific pages, but most parts will be identical for
  all platforms.}.


The platform dependent versions of these manuals are part of the respective
\RR{} installations. They can be downloaded as PDF files from the URL given
above, or directly browsed as HTML.

\begin{verbatim}
  http://cran.r-project.org/manuals.html
\end{verbatim}


The following manuals are available:


\begin{footnotesize}
  \begin{itemize}

    \item An Introduction to \RR{} is based on the former ``Notes on R'', gives an
      introduction to the language and how to use R for doing statistical analysis
      and graphics. \\
      %[browse HTML | download PDF ]

    \item A draft of The \RR{} language definition documents the language per se.
      That is, the objects that it works on, and the details of the expression 
      evaluation process, which are useful to know when programming \RR{} functions.\\
      %[browse HTML | download PDF ]

    \item Writing \RR{} Extensions covers how to create your own packages, write \RR{}
      help files, and the foreign language (C, C++, Fortran, \ldots) interfaces.\\
      %[browse HTML | download PDF ]

    \item R Data Import/Export describes the import and export facilities
      available either in \RR{} itself or via packages which are available from CRAN.\\
      %[browse HTML | download PDF ]

    \item R Installation and Administration.\\
      %[browse HTML | download PDF ]

    \item R Internals: a guide to the internal structures of \RR{} and coding
      standards for the core team working on R itself.\\
      %[browse HTML | download PDF ]

    \item The \RR{} Reference Index: contains all help files of the R standard and
      recommended packages in printable form, (approx. 3000 pages).\\
      %[download PDF, 15MB, approx. 3000 pages]

  \end{itemize}
\end{footnotesize}


The latex or texinfo sources of the latest version of these documents are contained
in every \RR{} source distribution. Take a look at the subdirectory doc/manual of the
extracted archive.


The HTML versions of the manuals are also part of most \RR{} installations. They are
accessible by using the function \Rfun{help.start}.
