
\section{Versions released in 2013 (10)}
\subsection*{3.0.2.7 (Dec/30/2013)}
\begin{itemize}
  \item The menu \texttt{View} was a bit improved.
  \item The \textit{User guide} has been revised.
\end{itemize}


\subsection*{3.0.2.6 (Dec/28/2013)}
\begin{itemize}
  \item \texttt{Tools}: the interface related to \texttt{Card} and \texttt{Mirrors} is now in a more logic place: \texttt{R}.
  \item The menu \texttt{View} was a bit improved. The more used options related to
  \texttt{Rgui}, \texttt{Tools} and \texttt{Rterm} were duplicated, but can be easily found.
  \item The menu \texttt{Help} was a bit enhanced and has a new option: \texttt{User list (discussion group)}.
  \item The \textit{User guide} has been revised.
\end{itemize}


\subsection*{3.0.2.5 (Dec/03/2013)}
\begin{itemize}
  \item Bug(s) fixed:
    \begin{itemize}
      \item Double click on \texttt{Tools/Database/Comments}.
      \item Double click on \texttt{Tools/Database/R mirros}.
      \item The content of \texttt{Help/Citation (put on clipboard)}.
      \item \texttt{Active line (BG)} option (\textit{Show/Hide}) and \texttt{Rterm}.
    \end{itemize}
  \item The default transparency was set to 0\%.
  \item Some small errors were fixed in the User guide.
  \item The version 3.0.2.3 and 3.0.2.4 were restricted to testers: many thanks to all testers,
   mainly to \texttt{Michal Sacharewicz} for his suggestions and tests related to \textit{R path identification}!
\end{itemize}


\subsection*{3.0.2.2 (Nov/20/2013)}
\begin{itemize}
  \item Bug(s) fixed:
    \begin{itemize}
      \item An intermitence related to \texttt{Sweave and Knitr}.
    \end{itemize}
  \item The \textit{User guide} has been revised.
  \item The main menu \textit{File} has a new option: \textit{Template}.
   This submenu has template to: \textit{R script}, \textit{R doc}
   (\textit{Function}, \textit{Dataset} and \textit{Empty}),
   \textit{R html}, \textit{R markdown} and \textit{R noweb}.
\end{itemize}


\subsection*{3.0.2.1 (Nov/19/2013)}
\begin{itemize}
  \item Bug(s) fixed:
    \begin{itemize}
      \item \texttt{Edit/Comment}.
      \item Recognition of latest version and update of all five database:
       \texttt{Shortcuts.xml}, \texttt{Completion.xml}, \texttt{Comments.xml}, \texttt{Rcard.xml}
       and \texttt{Rmirrors.xml}.
      \item \texttt{Options/Application} when the user choice \texttt{Cancel}.
    \end{itemize}
  \item The suport to \texttt{Knitr} package was improved and has more options and resources:
    \begin{itemize}
      \item A new button: \texttt{Knit to LaTeX (.Rnw)}.
      \item A new button: \texttt{Knit to HTML (.Rmd, Rhtml)}.
      \item Two new multi/complex highlighter: \texttt{R markdown} and \texttt{R html} were added.
      \item From now Tinn-R will always open/update the output \texttt{(*.tex, *.md and/or *.html)} file(s) after interpretation.
      \item If the option \texttt{Tools/Processing/Viewer/HTML} is marked,
       the \texttt{*.html} output file will be opened on the default browser.
    \end{itemize}
  \item Parts of the source code related to \texttt{Sweave} were enhanced.
   From now Tinn-R will always open/update the \texttt{.tex} file after interpretation.
\end{itemize}


\subsection*{3.0.1.10 (Nov/15/2013)}
\begin{itemize}
  \item Bug(s) fixed:
    \begin{itemize}
      \item \texttt{Rterm/Log/Gutter}.
    \end{itemize}
  \item Parts of the source code related to \texttt{R path} identification were enhanced.
  \item Parts of the source code related to \texttt{Code completion} were enhanced.
\end{itemize}


\subsection*{3.0.1.9 (Nov/07/2013)}
\begin{itemize}
  \item The URL of the project was changed to: http://nbcgib.uesc.br/lec/software/editores/tinn-r/en
  \item The \texttt{Rterm} interface now acepts the option \texttt{eoScrollPastEol}
   (Allows the cursor to go past the last character into the white space at the end of a line).
\end{itemize}


\subsection*{3.0.1.8 (Nov/06/2013)}
\begin{itemize}
  \item Bug(s) fixed:
    \begin{itemize}
      \item \texttt{Options/Highlighters (settings)} not remembering when more than
       one option (Bold, Italic and Underline) is marked.
      \item The visibility of the \texttt{Toolbar Edit}.
    \end{itemize}
  \item The \textit{User guide} has been revised. From this version it will be distributed
   only in PDF format and written entirely in \LaTeX.
  \item This version went through a long period of development and testing.
   This version brings several improvements to the project and is the
   best version ever released to the users. We hope you enjoy it!
  \item The versions 3.0.1.0, 3.0.1.1, 3.0.1.2, 3.0.1.3, 3.0.1.4, 3.0.1.5, 3.0.1.6 and 3.0.1.7
   were restricted to RC (Release Candidate) testers:
   many thanks to all testers, mainly to \texttt{Jakson A. Aquino} (main author of the nice Vim-R-plugin),
   for his good suggestions!
  \item The \textit{SynEdit} component used in Tinn-R project was updated to the latest version.
   Unlike previous versions, all necessary changes and adjustments in the sources of \textit{SynEdit}
   were made separately from the component, that is in units following the project source code of Tinn-R.
   This change is intended to facilitate the collective development of the Tinn-R sources.
  \item The sources of the project were deeply improved.
  \item Support to \textit{UNICODE} was added (this was a very hard work).
  \item The highlighters: \textit{R}, \textit{TeX}, \textit{All} and \textit{Text} were
   recreated to \textit{UNICODE} and all have new resources.
  \item The \texttt{Sweave} highlighter was renamed to \texttt{R noweb}. It is a multi highlighter: R + TeX.
  \item From this version Tinn-R Editor/GUI \texttt{needs a new version of the TinnR package:
   \textbf{TinnRcom}} distributed with the Tinn-R install/setup program.
  \item The \textit{RegEx PCRE} was implemented in the project.
   We have now more flexibility to deal with strings in the sources of the project.
  \item The window \texttt{About} was a bit enhanced.
  \item The support to \texttt{DDE protocol} was removed from the project, it is a bit
   old and hard to maintain.
  \item The database for \textit{call tip} was removed, due to constant updates of covered
   packages, it is hard to maintain.
  \item Two new database were added: \texttt{Rmirrors.xml} and \texttt{Comments.xml}.
   The first allows the user to manage the R mirrors and the second, the comments for all
   supported languages by Tinn-R.
  \item The \texttt{Comment}, \texttt{Uncomment first} and \texttt{Uncomment all}
   procedures were improved and from now are in the main menu \texttt{Edit}.
   Now it detects automatically the language and chunks (regions) for multiple (or complex) languages
   (like: \texttt{R noweb}, \texttt{R doc}, \texttt{HTML complex}, etc).
   The comments default are all now in the database \texttt{Comments.xml} (user configured).
   The user can force the use of an specific comment by unchecking the option
   \texttt{Tools/Database/Comments: (x) Auto detect language (recommended)}.
  \item The unique communication protocol with R is now the TCP/IP under the necessary
  \texttt{svSocket package}.
   There is a new option controlling whether R is connected and whether R is running automatically or not.
   By default it is set to True.
  \item The \texttt{R explorer} is almost fully based on temporary files, but the request is made silently
   (if the package svSoket is loaded and the communication (TCP/IP) is active).
   The reception of the information from R under TCP/IP was removed for this purpose.
   For massive exchanging information temporary files are faster, more accurate and stable.
  \item Parts of the source code, related to call tip and data completion were enhanced:
    it is now faster and more accurated (mainly related to the OOP).
  \item \texttt{Options/Application} window and related options were deeply upgraded.
    It is important to note that the editor options are now inside this interface.
  \item The menu \texttt{Options/Editor} was removed.
  \item The menu \texttt{View} has a new option: \texttt{Gutter (show/hide)} with the shortcuts
   \texttt{CTRL + ALT + G} associated to it. It will affect all instances of \texttt{SynEdit} class:
   Editor, Rterm and Log.
  \item The menu \texttt{Edit} and \texttt{Format} were a bit improved.
  \item An option menu \texttt{Encoding} was added to the main menu.
  \item The menu \texttt{Help} was a bit simplified.
  \item The main menu R was a bit improved and has three new options:
   \texttt{Set .trPaths (temporarily)}, \texttt{Get info (R and TinnRcom)} and \texttt{Update mirrors}.
  \item The button \texttt{R controls: packages} has new options: \texttt{Install} and
  \texttt{Load TinnRcom package}.
  \item The status bar was new panels showing the encoding and line endings of the files.
  \item The \texttt{Organize screen} procedure (related to screen arrangement of Tinn-R and Rgui.exe)
   from now runs twice: since just once is not enough for all flavors of the OS Windows.
  \item Inside \texttt{Tools/R explorer}(Tools window) now \texttt{CTRL + C} copy
   the R object name selected to the clipboard.
   This can be very useful.
\end{itemize}


\subsection*{2.4.1.7 (Mai/08/2013)}
\begin{itemize}
  \item Bug(s) fixed:
    \begin{itemize}
      \item \texttt{Copy} buttom of \texttt{Tools/Database (Rtip and Completion)}.
      \item \texttt{R send: line} (with empty files).
    \end{itemize}
  \item \texttt{Options/Application/R/Rterm/Options (Rterm)} has a new option:
   \texttt{Workspace image (close without ask for save)}.
   Thanks to \texttt{Roland E. Joss} for the suggestion.
  \item Some buttons of \texttt{R toolbar (print, plot, list names, list structure, edit and fix)}
   are now enabled only with at least one open file.
  \item The main menu \texttt{Web/R Editors/GUIs and IDEs/Tinn-R} was a bit enhanced.
   Tinn-R project now has its own \htmladdnormallink{Web page}{http://nbcgib.uesc.br/lec/software/editores/tinn-r/en}.
\end{itemize}


\subsection*{2.4.1.6 (Mai/06/2013)}
\begin{itemize}
  \item Bug(s) fixed:
    \begin{itemize}
      \item A bug, noticed after the release of the R 3.0.0, related to the \texttt{Options/Application/R/Path (R)}
       and recognition of the latest version has been fixed.
    \end{itemize}
  \item The user guide was revised: many thanks to \texttt{Jakson A. Aquino} (the main author of Vim-R-plugin).
\end{itemize}
