
\hypertarget{2009}{}
\section{Versions released in 2009 (16)}
\subsection*{2.3.4.3 (Dec/21/2009)}
\begin{itemize}
  \item Parts of the source code were optimized.
\end{itemize}


\subsection*{2.3.4.2 (Dec/18/2009)}
\begin{itemize}
  \item Bug(s) fixed:
    \begin{itemize}
      \item A bug related to the function \texttt{debug (package base)} in the
        Rterm interface was fixed. It was generating an exception
        \texttt{Access violation at address NUMBER in module 'Tinn-R.exe'.
          Read of address FFFFFFFF}. This bug started in version 2.3.4.0
        (Dec/06/2009).
    \end{itemize}
\end{itemize}


\subsection*{2.3.4.1 (Dec/13/2009)}
\begin{itemize}
  \item Bug(s) fixed:
    \begin{itemize}
      \item A bug related to \texttt{Search in Files} and its links in the
        \textit{Tools} panel was fixed. It was not going to the correct
        line in the file. This bug was related to the implementation of
        the resource \textit{Remember file state} in version 2.3.2.5
        (Nov/03/2009).
    \end{itemize}
  \item The user can now hide/show the button \texttt{Editor: current line to top}
    in the \textit{R toolbar}.
\end{itemize}


\subsection*{2.3.4.0 (Dec/06/2009)}
\begin{itemize}
  \item Bug(s) fixed:
    \begin{itemize}
      \item A bug related to the package \texttt{sem} by John Fox
        (functions: \texttt{read.moments} and \texttt{specify.model})
        which generated an error when submitting line by line within
        \textit{Rterm} interface was fixed. Many thanks to \texttt{Frank}
        for pointing it out!
    \end{itemize}
  \item Parts of the source code were optimized.
  \item The menu \textit{R/Rterm/Clear} and the pop-up menu Rterm
    (IO and Log), both related to \textit{Clear} option, were changed.
    This change allows more specific control over \textit{IO} and
    \textit{Log} of \textit{Rterm} interface.
  \item Two new options enable the user to send contiguous lines
    of script to the \RR{} interpreter.
  \item The small script generated by Tinn-R in the file Rprofile.site was
    changed:
    \begin{itemize}
      \item The parameter \texttt{dep=TRUE} was removed from the line

        \begin{Scode}
          install.packages(necessary[!installed], dep=TRUE)
        \end{Scode}

        since the \pkg{Hmisc} package, which enables Tinn-R to export
        \RR{} objects to TeX format, has several dependencies. This
        change will speed up the basic \RR{} configuration.

        \begin{Scode}
          ## Check necessary packages
          necessary <- c('TinnR', 'svSocket')
          installed <- necessary %in% installed.packages()[, 'Package']
          if (length(necessary[!installed]) >=1)
          install.packages(necessary[!installed])
        \end{Scode}

      \item Under the \textit{Rterm} interface the graphical menu should enable the
        user to choose the repository for a current session. However, this is
        inconsistent, sometimes showing this option and sometimes not. Therefore,
        the best option is to set the preferred repository from the file Rprofile.site.
    \end{itemize}
\end{itemize}


\subsection*{2.3.3.1 (Nov/10/2009)}
\begin{itemize}
  \item The menu \textit{Tools/Utils} was removed from the executable:
    it is restricted only to developers and we forgot to make it not
    visible.
\end{itemize}


\subsection*{2.3.3.0 (Nov/09/2009)}
\begin{itemize}
  \item Parts of the source code were optimized.
  \item The menu \textit{View} was enhanced with a new option \textit{Word wrap}.
    It allows the user to control the \textit{Editor}, \textit{Rterm/IO} and
    \textit{Rterm/Log} word wrap. Word wrap is a feature of most text editors,
    word processors, and web browsers, of breaking lines between and not within
    words, except when a single word is longer than a line.
\end{itemize}


\subsection*{2.3.2.6 (Nov/07/2009)}
\begin{itemize}
  \item Bug(s) fixed:
    \begin{itemize}
      \item A bug related to \textit{Rterm interface} and \texttt{Auto hide} option
        was fixed.
    \end{itemize}
  \item The automatic WordWrap option of Rterm interface is now off.
\end{itemize}


\subsection*{2.3.2.5 (Nov/03/2009)}
\begin{itemize}
  \item Bug(s) fixed:
    \begin{itemize}
      \item A bug related to \RR{} identification, if
        \textit{Options/Application/R/Path/Use latest installed version (always)}
        option is \textit{Yes} was fixed. This bug was detected only after the
        \RR{} 2.10.0.
    \end{itemize}
  \item Parts of the source code were optimized.
  \item Tinn-R now has the \textit{Remember file state} setting option. The
    file states are: all marks (0..9), the position of the cursor and top line
    of the file. It was implemented using a XML database (Cache.xml). It can
    be found at \textit{Options/Application/Main/General/Remember file state}.
  \item The option Send \textit{Marked block} was enhanced. Now it works as below:
    \begin{itemize}
      \item \textbf{The file has no marks}: the option will not be available
        (gray);
      \item \textbf{The file has one or more marks and the cursor is above
          the first mark, or below the last}: all text (above or below 
        this mark) will be sent, according to the cursor's position (above
        or below the mark);
      \item \textbf{The cursor is between any two adjacent marks}: all text
        between those two marks will be sent.
    \end{itemize}
  \item Under \textit{Rterm} interface it is possible to use the \texttt{TAB} as
    follow: \texttt{$>$ ba (followed by TAB)}. This procedure will
    send to \RR{} interpreter the instruction
    \texttt{apropos('\^{}bla', case.insensitive=FALSE)} and it will not be
    visible. \RR{} will returns a character vector giving the names of all
    objects in the search list matching \texttt{ba}. For example:

    \begin{Scode}
      > ba (followed by TAB)
      [1] "backsolve"             "backSpline"            "bacteria"
      [4] "balanceMethodsList"    "ballocation"           "bandwidth.kernel"
      [7] "bandwidth.nrd"         "barplot"               "barplot.default"
      [10] "bartlett.test"         "base.-.POSIXt"         "base.+.POSIXt"
      [13] "base.difftime"         "base.help"             "base.library"
      [16] "base.loadhistory"      "base.lockEnvironment"  "base.rbind.data.frame"
      [19] "base.save.image"       "base.savehistory"      "baseenv"
      [22] "basehaz"               "basename"

      > ba
    \end{Scode}

  \item The family \textit{rmControls} of components was removed from the
    project.
\end{itemize}


\subsection*{2.3.2.3 (Ago/06/2009)}
\begin{itemize}
  \item The \textit{Application options interface} was enhanced.
  \item Parts of the source code were optimized.
\end{itemize}


\subsection*{2.3.2.2 (Jul/20/2009)}
\begin{itemize}
  \item Bug(s) fixed:
    \begin{itemize}
      \item A bug related with \textit{Hotkeys (operational system)} when
        changing the status (Active not Active).
    \end{itemize}
\end{itemize}


\subsection*{2.3.2.1 (Jul/19/2009)}
\begin{itemize}
  \item Bug(s) fixed:
    \begin{itemize}
      \item An undesired and potentially danger option in the menu
        \textit{Tools/Utils}, used only in the development, was removed.
    \end{itemize}
\end{itemize}


\subsection*{2.3.2.0 (Jul/18/2009)}
\begin{itemize}
  \item Bug(s) fixed:
    \begin{itemize}
      \item All bugs related to database pointed out by users.
    \end{itemize}
\end{itemize}


\subsection*{2.3.1.0 (Jul/15/2009)}
\begin{itemize}
  \item Bug(s) fixed:
    \begin{itemize}
      \item All bugs related to database pointed out by users.
    \end{itemize}
\end{itemize}


\subsection*{2.3.0.0 (Jul/10/2009)}
\begin{itemize}
  \item Bug(s) fixed:
    \begin{itemize}
      \item The error message when typing \texttt{CTRL + TAB} inside
        the \textit{Rterm} interface whenever it was not split.
      \item It now remembers the position of the \textit{Tabs files}
        (top or botton) when starting.
      \item It now properly organizes the \textit{Tabs files} in
        relation to other toolbars whenever the user uses the
        \texttt{show/hide} option in the toolbars.
      \item The intermittency of completion resources.
      \item The \textit{Auto completion} and \textit{Data completion}
        now recognizes split by a dot as a complete word: for example
        \texttt{my.function(} and \texttt{my.data\$}.
      \item Under Windows Vista the option \textit{R/Configure/Permanent
          (Rprofile.site} now checks if the user has administrative
        privileges to change the content of the file \texttt{Rprofile.site},
        before inserting the necessary script. If the user receives an error
        message, it is necessary to manually change the security
        properties to enable \texttt{full control} of the folder
        \texttt{etc} where \RR{} is installed.
    \end{itemize}
  \item The \textit{Completion} resource migrated to XML, it is more flexible
    and easy to use. Now it is located in the the menu \textit{Tools/Database}.
  \item \textit{R card} and \textit{R tip} are now located in a more
    convenient place: \textit{Tool/Database}.
  \item The \textit{R tip} resource was updated.
  \item Menu \textit{Format/Block} was removed and all associated resources
    were relocated to a more logic place: \textit{Format/selection}.
  \item The \textit{User guide} was expanded/enhanced in various topics.
  \item The \textit{Application options interface} was enhanced.
  \item A new option in the \textit{Application options} allows more specific
    recognition of \texttt{Rgui}. Now it is possible to avoid any windows
    caption with the word \texttt{Console} to be recognized as a \RR{}
    instance.
  \item The interface \textit{Tinn-R hotkeys} was fully reworked and it is now
    more simple and efficient.
  \item Parts of the source code were optimized.
  \item A new resource was added to the \textit{R send}: \texttt{Clipboard}.
    It enables the user to send the content of the clipboard easily to Rterm.
  \item Sorry, due to bugs the highlighters \textit{Deplate} and \textit{Txt2tags}
    were removed from the project. New ones will be made in the future.
\end{itemize}


\subsection*{2.2.0.2 (Feb/09/2009)}
\begin{itemize}
  \item Bug(s) fixed:
    \begin{itemize}
      \item Using \texttt{TinnR package version 1.0.1 or 1.0.2} Tinn-R did starts
        when \RR{} starts with the instructions below in the Rprofile.site:

        \begin{Scode}
          \begin{verbatim}
            # uncomment the line below if you want Tinn-R starts always \RR{} starts
            options(IDE='C:/Tinn-R/bin/Tinn-R.exe')
          \end{verbatim}
        \end{Scode}

        The origin of this bug was the change of the packages \texttt{svIDE}
        (and others no longer necessary) to TinnR package. One function that
        should do the job was not present (\texttt{TinnR package version 1.0.1
          or 1.0.2}). In the \texttt{new version of the TinnR package (1.0.3)},
        this function \texttt{trStartIDE} was added. The option
        \texttt{R/Configure/Permanent (Rpfile.site)} will generate a new
        script:

        {\footnotesize                                                                         
          {\color {darkred}                                                                    
            \begin{verbatim}                                                                   
              ##===============================================================                
              ## Tinn-R: necessary packages and functions                                      
              ## Tinn-R: >= 2.2.0.2 with TinnR package >= 1.0.3                                
              ##===============================================================                
              ## Set the URL of the preferred repository, below some examples:                 
              options(repos='http://cran.at.r-project.org/')      # Austria/Wien               
              #options(repos='http://cran-r.c3sl.ufpr.br/')       # Brazil/PR                  
              #options(repos='http://cran.fiocruz.br/')           # Brazil/RJ                  
              #options(repos='http://www.vps.fmvz.usp.br/CRAN/')  # Brazil/SP                  
              #options(repos='http://brieger.esalq.usp.br/CRAN/') # Brazil/SP                  
                                                                                               
              library(utils)                                                                   
                                                                                               
              ## Check necessary packages                                                      
              necessary <- c('TinnR', 'svSocket')                                              
                                                                                               
              installed <- necessary %in% installed.packages()[, 'Package']                    
              if (length(necessary[!installed]) >=1)                                           
                install.packages(necessary[!installed])                                        
                                                                                               
              ## Load packages                                                                 
              library(TinnR)                                                                   
              library(svSocket)                                                                
                                                                                               
              ## Uncomment the two lines below if you want Tinn-R to always start R at start-up 
              ## (Observation: check the path of Tinn-R.exe)                                   
              #options(IDE='C:/Tinn-R/bin/Tinn-R.exe')                                         
              #trStartIDE()                                                                    
                                                                                               
              ## Option                                                                        
              options(use.DDE=T)                                                               
                                                                                               
              ## Start DDE                                                                     
              trDDEInstall()                                                                   
                                                                                               
              ## Short paths                                                                   
              .trPaths <- paste(paste(Sys.getenv('APPDATA'),                                   
                                      '\\Tinn-R\\tmp\\',
                                      sep=''),
                                c('',
                                  'search.txt',
                                  'objects.txt',
                                  'file.r',
                                  'selection.r',
                                  'block.r',
                                  'lines.r'),
                                sep='')
            \end{verbatim}
          } % color
        } % footnotesize

        If you uncomment this part:

        \begin{Scode}
          ## Uncomment the two lines below if you want Tinn-R starts always \RR{} starts
          ## (Observation: check the path of Tinn-R.exe)
          options(IDE='C:/Tinn-R/bin/Tinn-R.exe')
          trStartIDE()
        \end{Scode}

        The new TinnR package will do the job!
    \end{itemize}
\end{itemize}


\subsection*{2.2.0.1 (Feb/05/2009)}
\begin{itemize}
  \item Bug(s) fixed:
    \begin{itemize}
      \item Latex font \textit{Enphase} was fixed. It was inserting
        \textit{$\backslash$textbf\{\}} instead of
        \textit{$\backslash$emph\{\}}.
      \item Save and load workspace is now enabled only if Rterm is
        running.
      \item A bug associated with the \textit{Color preferences interface}
        related to \textit{Txt2tags} and \textit{Deplate} syntax, not
        allowing to change the background color of the root element in
        the correct way was fixed.
    \end{itemize}
  \item All prior documentation of the Tinn-R project was updated. Some
    parts were expanded and others were excluded, new ones were generated and,
    finally, all were joined in the new \textit{User guide}. This
    \textit{What is new} is now part of this user guide. The source
    files (written in Txt2tags) of this User guide are available in the
    folder \texttt{doc/english/user\_guide} where Tinn-R is installed.
    This way, we hope that the user will be able to help us in making
    it better, day by day, by sending us any useful contribution.
  \item The \textit{R toolbar} can now be docked at left, top, right and
    bottom side of the main interface. Some issues related to \textit{Rterm}
    and \textit{Tools} interface when dragging the \RR{} tools bar, have
    not completely been solved yet. In order to fix any problem, hide and
    show again the \textit{Rterm} or \textit{Tools} interface.
  \item The match brackets resource (default shortcut is \texttt{CTRL + B})
    was added also to the Rterm interface (\textit{IO} and \textit{Log}).
  \item The \textit{Tools} interface was a bit reworked and the menu
    \textit{Views} was changed to accommodate the changes.
  \item The menus \textit{View} and \textit{Help} were a bit reworked.
  \item The send and control \RR{} resources were extended to all
    instances of the synEdit class. In other words, if you put the
    cursor in any word, or select any text in the \textit{Editor}
    (split or not), \textit{IO} or \textit{Log} (docked or not, in the
    same or distinct monitors) interface and, after this, select an
    action (print content, plot, etc) it will be executed.
\end{itemize}
