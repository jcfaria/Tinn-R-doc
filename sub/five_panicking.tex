
\section{I'm panicking: what do I do?}
\index{secrets!I'm panicking}

First: \textbf{don't panic (it's not good for your health)}!

\begin{itemize}
\item \textbf{Do not uninstall and reinstall Tinn-R or \RR{} unnecessarily}.
  The origin of the problem (although rare) can be mainly the folder where Tinn-R has installed its ini files
  (it does not use the windows registry). For some reason, it was corrupted or damaged.
  The first step is to know where Tinn-R stores these files: \texttt{Help/Ini files (path information)}
  will do the job for you. After that, close Tinn-R,
  rename (or remove) this folder (it will be recreated after Tinn-R is restarted).
  If this folder is not visible in your computer \textit{\htmladdnormallink{see useful links here}{\#faq\_trpaths}}.
\item \textbf{Feel free to write to the \textit{\htmladdnormallink{project leader}{mailto:joseclaudio.faria@gmail.com}}}.
  If you submit a bug report, please provide as much detail as possible.
  This includes indicating the Tinn-R version, your operating system (Windows XP, Windows 7, etc) and
  language (English, French, Portuguese). If the bug is related to an interface with \RR{},
  indicate which version of \RR{} you are using, as well as whether you are running Rterm or Rgui.
  You should also add the content of the \textit{Tools/Results/Ini log}
  interface since this will help us to address the issue promptly.
\end{itemize}
