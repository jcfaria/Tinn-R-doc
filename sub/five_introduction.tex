
\section{Introduction}
\index{secrets!introduction}

Let us assume that you are a basic user of \RR{} and you do not understand the intricacies of computer
languages and that you want to write and run an R-script.
The software providing an interface between \RR{} and the user is called a \textit{user interface}.
If this interface also has graphical capabilities, it is called a graphical user interface or \textit{GUI}.

\RR{} running under Windows and Mac have a graphical interface (Rgui) that allows you to
submit your commands and see the respective results, but that interface is a bit limited.

\textit{\href{http://www.sciviews.org/\_rgui/}{Many attempts}} have been made to provide \RR{}
with a user friendly graphical interface running on Windows. One of the most successful ones is Tinn-R,
which is arguably the most used "GUI" among \RR{} users on Windows.

A large number of Tinn-R users do not know how to improve its performance and their productivity,
so this chapter is an attempt to help you to make the most out of its features.

Let us start, then. After downloading Tinn-R from its main
\textit{\href{https://tinn-r.org/en/}{Web page}}
or
\textit{\href{http://sourceforge.net/projects/tinn-r/}{Sourceforge}},
you should install and open the software.
