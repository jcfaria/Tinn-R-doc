
\hypertarget{2021}{}
\section{Versions released in 2021 (06)}
\index{What is New?!2021}

\subsection*{8.1.2.6 (apr/23/2021)}
\begin{itemize}
  \item The main menu \texttt{main menu: R/Control/Packages/Install}
     has been changed, it now installs the selected package in the editor
     (or \texttt{Rterm} interface).
  \item A new button has been created on the \texttt{Rterm} interface task bar that enables the user
    to send or resend any line or selection. This feature (using \texttt{CTRL + ENTER}) has been available for
    a long time on the \texttt{Rterm} interface, but not in the form of a button. We believe that the button
    makes it easier to use for new users. \texttt{Rterm: send or resend (any line or selection) CTRL + ENTER}.
  \item Some adjustments were made to the license request mechanism.
  \item Some improvements have been made to some program windows.
  \item The license verification and validation procedure is still disabled in this version.
    So, if you want, you can request a license to help improve the mechanism,
    as it will be being tested on more diverse hardware and software platforms than
    those used by testers. In the next version, the validation mechanism will be enabled.
\end{itemize}

\subsection*{8.1.2.5 (apr/21/2021)}
\begin{itemize}
  \item Bug(s) fixed:
    \begin{itemize}
      \item Bugs related to the assignment of some \texttt{Hotkeys (R)}, which required restarting
        Tinn-R to function, have been fixed.
    \end{itemize}
  \item From that version on, the Tinn-R program has version control.
    The Tinn-R project remains free to use for the educational sector
    at any level. For these, request a free license (renewed annually).
    See \texttt{(main menu: Help/About/Licensing)} and \texttt{(main menu: Help/License manager)}.
    The license verification and validation procedure is disabled in this version.
    So, if you want, you can request a license to help improve the mechanism,
    as it will be being tested on more diverse hardware and software platforms than
    those used by testers. In the next version, the validation mechanism will be enabled.
\end{itemize}

\subsection*{8.1.2.4 (apr/18/2021) - Restrict to testers}
\begin{itemize}
  \item New options to get and set the R's working directory.
  \item Improvement of the mechanism for locating particular (simple and complex) words in the R environment.
    The new mechanism is heavily based on regular expressions, so unexpected things can happen in unforeseen
    cases as well as in long comments on lines that contain the desired word (associated with a data function).
    If something unexpected happens: select the desired word and report the error to the developers.
\end{itemize}

\subsection*{8.1.2.3 (mar/26/2021) - Restrict to testers}
\begin{itemize}
  \item Bug(s) fixed:
    \begin{itemize}
      \item \texttt{Check for update} procedure. The bug was due to the change of the project server and
        security rules of the communication protocol.
    \end{itemize}
\end{itemize}

\subsection*{8.1.2.2 (mar/25/2021) - Restrict to testers}
\begin{itemize}
  \item Some improvements and refinements have been made in the previous version 8.1.2.1.
\end{itemize}

\subsection*{8.1.2.1 (jan/01/2021) - Restrict to testers}
\begin{itemize}
  \item Bug(s) fixed:
    \begin{itemize}
      \item Black streaks on the main interface when the application is closed maximized. Not yet
        it was possible to find the source of the bug, so to work around the problem, if it is maximized
        when closed, the program will automatically leave the maximized logic state. Therefore,
        when reopened, the main window will be displayed in the stable size and position before being maximized and closed.
       \item Tools/Markup/LaTeX: the taskbar buttons were missing their options. The bug has been fixed.
    \end{itemize}
  \item Implementation of version control.
    The Tinn-R project remains free to use for the educational sector
    at any level. For these, request a free license (renewed annually).
    See \texttt{(main menu: Help/About/Licensing)} and \texttt{(main menu: Help/License manager)}.
\end{itemize}
