
\hypertarget{2021}{}
\section{Versions released in 2021 (15)}
\index{What is New?!2021}

\subsection*{8.1.3.5 (jun/22/2021)}
\begin{itemize}
  \item Bug(s) fixed: 
    \begin{itemize}
      \item A bug associated to the \texttt{completion resource} with content longer than one line 
        (eg: \texttt{ma}, \texttt{iec}, etc)
        has been fixed. This bug was accidentally introduced in previous version 8.1.3.4.
      \item A bug associated to the interface \texttt{Tools/R/Explorer} and the \texttt{package dplyr} was fixed.
      \item A bug associated to error message \texttt{baseenv()} from R flavors MRO and R-devel was fixed.
        This bug was associated to the package svMisc. Thanks to SciViews Team.
    \end{itemize}
  \item The toolbar of the \texttt{Tools/Project} interface has been redesigned.
  \item The \texttt{Tools/R/Explorer} interface has the default option, from now on, not to display all objects.
    That is, hidden objects from the environment (\texttt{.object}) are not shown.
  \item A new updated version (1.0.22) of the \texttt{TinnRcom} package is being distributed with this version 
    of the project.
  \item The license verification and validation procedure is still disabled in this version.
    So, if you want, you can request a license to help improve the mechanism,
    as it will be being tested on more diverse hardware and software platforms than
    those used by testers. In the future version, the validation mechanism will be enabled.
\end{itemize}

\subsection*{8.1.3.4 (jun/01/2021)}
\begin{itemize}
  \item The project's old web page (\texttt{https://nbcgib.uesc.br/tinnr/en/}) has been permanently
    removed.
  \item The demo project files were updated.
  \item The files in the \texttt{utils} folder have been updated. As a result, the versions of
    packages that the TinnRcom package depends on are checked at R startup and will be updated
    whenever necessary.
  \item Balthazar Mattos Farnese started working on the project as a contributor responsible
    for the project's web page and email server. Welcome Balthazar.
  \item Some improvements and refinements have been made in the previous version 8.1.3.3.
		\item Automatic indentation (after carriage return) during editing has been considerably
    improved in this version; it was hard work. If you notice any inconsistencies,
    please prepare a small sample file and send detailed information to be corrected.
  \item The license verification and validation procedure is still disabled in this version.
    So, if you want, you can request a license to help improve the mechanism,
    as it will be being tested on more diverse hardware and software platforms than
    those used by testers. In the future version, the validation mechanism will be enabled.
\end{itemize}

\subsection*{8.1.3.3 (mai/10/2021)}
\begin{itemize}
  \item Improvements were made to the \texttt{Tools/Markup/LaTeX} interface.
  \item Improvements have been made to the \texttt{demo project}.
  \item Improvements have been made to the recently implemented features that enable you to
    mark a \LaTeX ~file (\texttt{.tex}) as the main (root), both in the main interface and in
    the projects.
  \item The license verification and validation procedure is still disabled in this version.
    So, if you want, you can request a license to help improve the mechanism,
    as it will be being tested on more diverse hardware and software platforms than
    those used by testers. In the future version, the validation mechanism will be enabled.
\end{itemize}

\subsection*{8.1.3.2 (mai/09/2021)}
\begin{itemize}
  \item Bug(s) fixed:
    \begin{itemize}
      \item Bugs associated with some templates were fixed.
    \end{itemize}
  \item The license verification and validation procedure is still disabled in this version.
    So, if you want, you can request a license to help improve the mechanism,
    as it will be being tested on more diverse hardware and software platforms than
    those used by testers. In the future version, the validation mechanism will be enabled.
\end{itemize}

\subsection*{8.1.3.1 (mai/08/2021)}
\begin{itemize}
  \item Bug(s) fixed:
    \begin{itemize}
      \item A bug associated with automatic updating (\texttt{menu/Help/Check for update}) was fixed. 
        The origin of the bug was the exchange of the NBCGIB project server for Digital Ocean and the 
        differentiated security standards.
      \item A bug associated with the position of the code completion (or tips) window when two monitors
        are used has been fixed.
    \end{itemize}
  \item All dialogs for choosing folders (Updater, R choose dir, etc.) have been improved 
    due to the use of the most modern Windows API.
  \item The elements of the \texttt{Tools/LaTeX} interface have been updated and improved.
    The \texttt{LaTeX\_symbols.tex} file was added to the demo project and the corresponding
    \texttt{LaTeX\_symbols.pdf} file was added to the \textbf{doc} folder where the application
    was installed.
  \item The license verification and validation procedure is still disabled in this version.
    So, if you want, you can request a license to help improve the mechanism,
    as it will be being tested on more diverse hardware and software platforms than
    those used by testers. In the future version, the validation mechanism will be enabled.
\end{itemize}

\subsection*{8.1.2.10 (mai/03/2021)}
\begin{itemize}
  \item A new option has been added that enables the user  
    \textbf{to sets the main \LaTeX ~file}. Thus, he can be editing in any
    file that, when compiling (generate the PDF or DVI), the instruction will be sent correctly
    for the compiler. This new feature speeds up the process of creating files (PDF and DVI) 
    as the user does not need to navigate to the main file each time to submit 
    to the compilation. The feature is functional for all options
    associated with compilation (.pdf, .dvi, makeindex and bibitex). This resource is volatile
    with each session and project, that is, it is not stored with the project information.
  \item When there are two main file indications for \LaTeX ~(one in the \textbf{project} and the
    other in the \textbf{main interface} for any file), this option, from within the project, 
    takes precedence, that is, it is mandatory.
  \item The license verification and validation procedure is still disabled in this version.
    So, if you want, you can request a license to help improve the mechanism,
    as it will be being tested on more diverse hardware and software platforms than
    those used by testers. In the future version, the validation mechanism will be enabled.
\end{itemize}

\subsection*{8.1.2.9 (mai/02/2021)}
\begin{itemize}
  \item Tinn-R is enabled (for many years) to do reverse search in DVI and PDF. 
    However, only the reverse search in DVI was documented, as it had been used 
    in large projects (books) written in \LaTeX. ~Recently working on a book directly
    from \LaTeX ~to PDF, the feature was tested and documented in the user guide.
    The feature is fully functional as long as the correct directive is used: 
    \textbf{pdflatex -c-style-errors --synctex=1}.
  \item A new project-related option has been added that enables the user  
    \textbf{to sets the main \LaTeX ~project file}. Thus, he can be editing in any project
    file that, when compiling (generate the PDF or DVI), the instruction will be sent correctly
    for the compiler. This new feature speeds up the process of creating files (PDF and DVI) 
    as the user does not need to navigate to the main file each time to submit the project 
    to the compilation. For projects with many files, this feature (already present in more 
    specialized \LaTeX ~editors) makes work much easier. The feature is functional for all options
    associated with compilation (.pdf, .dvi, makeindex and bibitex). This resource is volatile
    with each session and project, that is, it is not stored with the project information.
  \item The license verification and validation procedure is still disabled in this version.
    So, if you want, you can request a license to help improve the mechanism,
    as it will be being tested on more diverse hardware and software platforms than
    those used by testers. In the future version, the validation mechanism will be enabled.
\end{itemize}

\subsection*{8.1.2.8 (apr/28/2021)}
\begin{itemize}
  \item Bug(s) fixed:
    \begin{itemize}
      \item Version 8.1.2.7, due to the detection of two serious bugs, accidentally
        inserted in the project's source code, has been removed from all project repositories.
        The bugs have been fixed. \color{darkred}{It is strongly recommended not to use version 8.1.2.7!}
    \end{itemize}
  \item Some improvements have been made to some program windows.
  \item Automatic indentation (after carriage return) during editing has been considerably
    improved in this version; it was hard work. If you notice any inconsistencies,
    please prepare a small sample file and send detailed information to be corrected.
  \item The license verification and validation procedure is still disabled in this version.
    So, if you want, you can request a license to help improve the mechanism,
    as it will be being tested on more diverse hardware and software platforms than
    those used by testers. In the future version, the validation mechanism will be enabled.
\end{itemize}

\subsection*{8.1.2.7 (apr/26/2021) - \color{darkred}{Removed from all project repositories}}
\begin{itemize}
  \item Some improvements have been made to some program windows.
  \item Some default database were updated (\texttt{Rmirrors.xml}, \texttt{Completion.xml}).
  \item Trying to circumvent the technological delay of the Brazilian economy related to
    the difficulties of transferring values from abroad to Brazil, the project now receives
    donations via cryptocurrencies: \textbf{Bitcoin} and \textbf{Bitcoin Cash}.
    In the main menu Help/About/Donation, the accounts for receiving donations (transfers)
    can be copied \texttt{(CTRL + C)} and pasted \texttt{(CTRL + V)} at the destination.
  \item The license verification and validation procedure is still disabled in this version.
    So, if you want, you can request a license to help improve the mechanism,
    as it will be being tested on more diverse hardware and software platforms than
    those used by testers. In the future version, the validation mechanism will be enabled.
\end{itemize}

\subsection*{8.1.2.6 (apr/23/2021)}
\begin{itemize}
  \item The main menu \texttt{main menu: R/Control/Packages/Install}
     has been changed, it now installs the selected package in the editor
     (or \texttt{Rterm} interface).
  \item A new button has been created on the \texttt{Rterm} interface task bar that enables the user
    to send or resend any line or selection. This feature (using \texttt{CTRL + ENTER}) has been available for
    a long time on the \texttt{Rterm} interface, but not in the form of a button. We believe that the button
    makes it easier to use for new users. \texttt{Rterm: send or resend (any line or selection) CTRL + ENTER}.
  \item Some adjustments were made to the license request mechanism.
  \item Some improvements have been made to some program windows.
  \item The license verification and validation procedure is still disabled in this version.
    So, if you want, you can request a license to help improve the mechanism,
    as it will be being tested on more diverse hardware and software platforms than
    those used by testers. In the future version, the validation mechanism will be enabled.
\end{itemize}

\subsection*{8.1.2.5 (apr/21/2021)}
\begin{itemize}
  \item Bug(s) fixed:
    \begin{itemize}
      \item Bugs related to the assignment of some \texttt{Hotkeys (R)}, which required restarting
        Tinn-R to function, have been fixed.
    \end{itemize}
  \item From that version on, the Tinn-R program has version control.
    The Tinn-R project remains free to use for the educational sector
    at any level. For these, request a free license (renewed annually).
    See \texttt{(main menu: Help/About/Licensing)} and \texttt{(main menu: Help/License manager)}.
    The license verification and validation procedure is disabled in this version.
    So, if you want, you can request a license to help improve the mechanism,
    as it will be being tested on more diverse hardware and software platforms than
    those used by testers. In the future version, the validation mechanism will be enabled.
\end{itemize}

\subsection*{8.1.2.4 (apr/18/2021) - Restrict to testers}
\begin{itemize}
  \item New options to get and set the R's working directory.
  \item Improvement of the mechanism for locating particular (simple and complex) words in the R environment.
    The new mechanism is heavily based on regular expressions, so unexpected things can happen in unforeseen
    cases as well as in long comments on lines that contain the desired word (associated with a data function).
    If something unexpected happens: select the desired word and report the error to the developers.
\end{itemize}

\subsection*{8.1.2.3 (mar/26/2021) - Restrict to testers}
\begin{itemize}
  \item Bug(s) fixed:
    \begin{itemize}
      \item \texttt{Check for update} procedure. The bug was due to the change of the project server and
        security rules of the communication protocol.
    \end{itemize}
\end{itemize}

\subsection*{8.1.2.2 (mar/25/2021) - Restrict to testers}
\begin{itemize}
  \item Some improvements and refinements have been made in the previous version 8.1.2.1.
\end{itemize}

\subsection*{8.1.2.1 (jan/01/2021) - Restrict to testers}
\begin{itemize}
  \item Bug(s) fixed:
    \begin{itemize}
      \item Black streaks on the main interface when the application is closed maximized. Not yet
        it was possible to find the source of the bug, so to work around the problem, if it is maximized
        when closed, the program will automatically leave the maximized logic state. Therefore,
        when reopened, the main window will be displayed in the stable size and position before being maximized and closed.
       \item Tools/Markup/LaTeX: the taskbar buttons were missing their options. The bug has been fixed.
    \end{itemize}
  \item Implementation of version control.
    The Tinn-R project remains free to use for the educational sector
    at any level. For these, request a free license (renewed annually).
    See \texttt{(main menu: Help/About/Licensing)} and \texttt{(main menu: Help/License manager)}.
\end{itemize}
