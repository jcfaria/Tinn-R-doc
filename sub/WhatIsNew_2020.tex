
\hypertarget{2020}{}
\section{Versions released in 2020 (05)}
\index{What is New?!2020}

\subsection*{6.1.1.6 (mai/09/2020)}
\begin{itemize}
  \item Bug(s) fixed:
    \begin{itemize}
      \item All dependencies of the TinnRcom package that are distributed with the Tinn-R setup
        (formatR, svMisc and svSocket), as well as the TinnRcom package itself, have been rebuilt
        with the R-devel version (4.1). The startup issues that arose in the current version 4.0 of
        \RR~ have been fixed.
      \item A bug associated with the package \texttt{semPlot} was fixed.
        Thanks to \texttt{Frank} for pointing it out.
    \end{itemize}
  \item Filter of \texttt{Tools/R/Mirros} from now on will be applied by \texttt{Host}.
\end{itemize}

\subsection*{6.1.1.5 (jan/06/2020)}
\begin{itemize}
  \item Some improvements and refinements have been made in the previous version 6.1.1.4.
\end{itemize}

\subsection*{6.1.1.4 (jan/02/2020)}
\begin{itemize}
  \item Many improvements and refinements have been made in the previous version 6.1.1.3.
\end{itemize}

\subsection*{6.1.1.3 (jan/02/2020)}
\begin{itemize}
  \item Some parts of the Object Pascal \textbf{pipe} were rewritten. It is now faster, stable and robust.
  \item The handling of all shortcuts, keystrokes and hotkeys are now centered on a single user interface
    that is simpler and more efficient than previous options. As result, a new menu option has been
    created: \texttt{Options/Shortcuts/keystrokes/hotkeys (map)}. It was a very hard and time consuming work,
    we hope you enjoy the final results.
  \item We are looking for the maximum freedom and efficiency to interact with the R interpreter.
    Thus, improvements have been made to the \texttt{Rterm IO} interface:
    \begin{itemize}
      \item From now on, to send any prior line (or just any selection) to the R interpreter, you must press the
        \textbf{CTRL + ENTER} shortcut keys (not configurable).
      \item In the latest line the user can use: a simple \textbf{ENTER} or \textbf{CTRL + ENTER} to send the current line.
      \item The \texttt{R history} can be visualized (and filtered progressively as you type) in a new \texttt{SynEdit completion window}.
        The shortcut is \textbf{CTRL + ALT + SPACE} (not configurable).
    \end{itemize}
  \item From now on, the shortcut \textbf{CTRL + ENTER} (not configurable) is also associated to send line (or selection) from the
    Editor to the \RR{} interpreter. The previous use (send the current line and insert a new one) was removed. This was an old request
    from users but we believe it is no longer used much. Therefore, unless otherwise stated, we don't wish wish to make this feature available anymore.
  \item The option to send the clipboard content, with shortcut \textbf{CTRL + Q} by default, to \RR{} interpreter was removed.
  \item The \texttt{R highlighter} was improved, it has new objects and from now on, will be case sensitive. That is, it will highlight, for example,
    \textbf{LETTERS, letters, CO2, co2, \ldots}, recognized and different \RR~ objects, in the same group with the same attributes.
  \item From this version, the instructions stored in \texttt{R history} are only those typed in and submitted directly into \texttt{Rterm IO} interface.
  \item The highlighter identifier \texttt{Note} was renamed to \texttt{Note\_0}. It makes more sense, since \texttt{Note\_1} and \texttt{Note\_2} were created
    in previous version.
  \item Due to low usage, all shortcuts (default) associated with Txt2tags to Deplate converters have been removed.
  \item A new shortcut (default) has been associated with the option \texttt{Save file as: SHIFT+CTRL+S}.
  \item The interface \texttt{Highlighters setting} was improved. From now on, the program will be distributed with at least 4 fast coloring sets:
    Default, Dark, Gray and LGray. Additionally, the interface allows you to restore all default values with a new option.
\end{itemize}
