
\hypertarget{2022}{}
\section{Versions released in 2022 (02)}
\index{What is New?!2022}

\subsection*{8.2.2.1 (jul/06/2022)}
\begin{itemize}
  \item The \texttt{web} menu item received updates and improvements.
  \item The license verification and validation procedure is still disabled in this version.
    So, if you want, you can request a license to help improve the mechanism,
    as it will be being tested on more diverse hardware and software platforms than
    those used by testers. In the future version, the validation mechanism will be enabled.
\end{itemize}

\subsection*{8.2.1.1 (jul/02/2022)}
\begin{itemize}
  \item Bug(s) fixed:
    \begin{itemize}
        \item The issues that emerged with the new version of R (4.2.0 - UTF-8) have been fixed.
          Because the Tinn-R project supports both old and new versions of R, \textbf{from now on Rgui.exe must be launched from within Tinn-R so that it is properly
          identified at startup}.
      \item A bug associated to MRO, the \texttt{Rterm interface} and the \texttt{delete\_checkpoint()} function of the \texttt{package checkpoint} was fixed.
    \end{itemize}
  \item Increase (\texttt{CTRL + Mouse Reel UP}) and decrease (\texttt{CTRL + Mouse Reel DOWN}) of editor font and Rterm via CTRL + Mouse Reel.
  \item Almost all of the interpreter's control options were enabled when there is no open file. In this case searches for words or selections will be done
    in the instance of SynEdit Rterm/IO. This allows greater freedom and interactivity of the Rterm interface.
  \item Returning focus from other applications to Tinn-R (Editor or Rterm) has been improved.
  \item Some default \texttt{R Hotkeys} have been changed and some new ones have been defined.
  \item In the project interface double clicking opens only individual files, it has been disabled for groups and entire project. For entire project and groups
    the drag option remains functional.
  \item The license verification and validation procedure is still disabled in this version.
    So, if you want, you can request a license to help improve the mechanism,
    as it will be being tested on more diverse hardware and software platforms than
    those used by testers. In the future version, the validation mechanism will be enabled.
\end{itemize}
