
\hypertarget{2019}{}
\section{Versions released in 2019 (04)}
\index{What is New?!2019}

\subsection*{5.3.4.1 (apr/22/2019)}
\begin{itemize}
  \item Two new identifiers have been added to highlighters
�� \textbf{R} and \textbf{Text}: \texttt{Note\_1} and \texttt{Note\_2}.
�� They are useful to be used as differentiated marks (now in three levels) in scripts and texts.
\end{itemize}

\subsection*{5.3.3.1 (fev/25/2019)}
\begin{itemize}
  \item Bug(s) fixed:
    \begin{itemize}
      \item A bug associated with the procedure \texttt{Help/Check for update} and the prior
        \texttt{5.3.2.1 (fev/12/2019) portable} version was fixed.
    \end{itemize}
\end{itemize}

\subsection*{5.3.2.1 (fev/12/2019)}
\begin{itemize}
  \item The recognition of complex R object names for \texttt{print}, \texttt{plot}, \texttt{names}
    \texttt{structure}, \texttt{edit}, \texttt{fix}, \texttt{help}, \texttt{example} and \texttt{open example}
      were improved. For example, for all cases:
    \begin{itemize}
      \item iris
      \item iris[1]
      \item iris[c(1, 2)]
      \item iris3(1, 1, 3)
      \item iris\$Sepal.Length
      \item iris['Sepal.Length']
      \item iris[["Sepal.Length"]]
    \end{itemize}
    that it will be correctly recognized.
  \item The \texttt{Replace} procedure was improved. From herefter, it will remember, for all files, the same
    word list used for replacement.
  \item The \htmladdnormallink{Tinn-R web site}{https://nbcgib.uesc.br/tinnr/en/} was updated,
    the new URL is: \textbf{https://nbcgib.uesc.br/tinnr/en/}.
\end{itemize}

\subsection*{5.3.1.1 (jan/06/2019)}
\begin{itemize}
  \item The status bar was redesigned.
  \item The option \texttt{R send: line} has a new feature that allows the main editor to send sequential instructions
    while the interpreter returns the signal +(plus) after each incomplete instruction. This option can be fast and easily switched in the status bar.
  \item A new option to send instructions to R interpreter is being implemented: \texttt{R send: smart}.
    The goal is to simplify it for beginner users to send complete instruction blocks when the cursor is located in a complex context:
    out of the first line of functions, instructions with several contiguous lines and similar situations.
    The work is underway, but not yet fully done. Because of that, this feature is disabled. If the user try to use/enable it, he/she will get the message:
    \texttt{This feature is still in development.}
  \item The interface \texttt{Hotkeys (operational system)} was simplified and improved. Additionally, the number of customizable user actions increased to 20.
   On demand, this number can be easily increased. Due to these changes, the previous hotkeys will be lost, recreating them from the new GUI is necessary.
  \item Excepting the source files of Project Description (\texttt{Tinn\_R.dpr} and \texttt{Tinn-R\_portable.dpr}),
   all source code files were unified. Therefore, from now on it became easier to compile and distribute the
   \texttt{setup} and \texttt{portable} \texttt{Tinn-R\_X.X.X.X\_setup.exe} and \\
   \texttt{Tinn-R\_X.X.X.X\_portable.zip} versions, respectively.
   The custom support in PortableApps standards was completely removed.
  \item This version allows using some special characters
    (for example: \textbf{.aa}, \textbf{-bb}, \textbf{\_cc}, \textbf{~dd}, \textbf{?ee} \textbf{!ff} and \textbf{@gg}) as trigger to data completion.
    All triggers of the new HTML group begin with \textbf{'<'} character. It will give more flexibility to create completion to other languages.
  \item The mechanism of completion was enhanced a bit. From now on it will be more user-friendly regarding to the indentation of the text in all supported languages.
  \item Corrections were made in the popup menu interfaces \texttt{Rterm} \texttt{IO} and \texttt{LOG}.
  \item The option to donate to the project has changed. Hereafter, it will be a bank account in the name of the project leader.
    \textbf{Please, do not use the PayPal account anymore}.
\end{itemize}