
\pagestyle{Editus-mainmatter}


\chapter*{Preface}
\addcontentsline{toc}{chapter}{Preface}
\addcontentsline{toc}{section}{Bill Venables}

\RR{} is a software environment for data analysis and graphics that provides an
implementation of the S language of John Chambers. It is a free software, and
in recent years it has grown enormously in popularity all over the data analysis
world, and even wider. The original system was written by two New Zealand
statisticians, Ross Ihaka and Robert Gentleman, who happened also to be
interested in software engineering. In the early 90's the only platform
they had on which to teach statistics was Apple Macintosh, which at the
time had almost no suitable statistical software available. Their solution
was to implement a version of the S language using a Scheme interpreter they
had written essentially as a programming exercise.  And \RR{} was born.
Since its public release in 1993 it has benefited enormously from the
programming contributions of developers and users all over the world, who
in turn have benefited enormously from \RR{}.\\

Although \RR{} can be used directly at the command line, to use it effectively
does ultimately require some form of script editor with a connection to the
\RR{} system. In fact the more supports the editor can give you, the easier
using \RR{} becomes. Features like colour highlighting of syntax, clear demarcation
of comments and easy facilities for indenting code to reveal the underlying
structure, although irrelevant to \RR{} itself, are of immense benefit to the user.
On Windows simple editors such as Notepad, or even the inbuilt script editor
that now comes as part of \RR{} itself, while adequate for very simple tasks,
become increasingly inadequate for \RR{} projects of any real size or complexity.
The birth of Tinn-R has some curious parallels with the birth of \RR{} itself.
Jos� Cl�udio Faria wrote the original version, based on the existing Tinn
editor, for his own personal use. Colleagues and students soon became aware
of the initiative and began not only using it, but in some cases contributing
to its development. With a generosity now typical of most people in the \RR{}
community, Jos� Cl�udio released the system under the GPL2 (or later) public
license for all to enjoy. Now the system is widely followed all around
the world in the Windows \RR{} community.\\

Tinn-R provides not only an \RR{}-aware editor and submission process to the system,
but a comprehensive project management system as well, including editing
facilities for many types of file other than \RR{} scripts. Although most users
would begin using it as a script editor for \RR{} alone, as they become familiar
with the system, again somewhat like \RR{} itself, there always seems to be some
further useful feature waiting to be discovered. The present e-book will
hopefully expedite this discovery phase, but essentially users do need to
use the system as they uncover its scope, as well as prompt the process by
reading about it.\\

I warmly congratulate Jos� Cl�udio and his team on a very polished and highly
useful contribution and sincerely thank them for their generosity in releasing
it. I am very sure the whole \RR{} community heartily agrees.\\

\begin{flushright}\noindent
  Bill Venables\\
  Australia\\ 
  14 November 2010
\end{flushright}



\newpage
\section*{Preface from the Rmetrics Editor - 1. ed.}
\addcontentsline{toc}{section}{Rmetrics (1. ed.)}

It is a pleasure to introduce the first book in the R/Rmetrics series not
authored by the Rmetrics team.\\

This book, by Jos� Cl�udio Faria, Philippe Grosjean, Enio
Galinkin Jelihovschi and Ricardo Pietrobon, describes the Tinn-R editor,
a very powerful code editor for \RR{}. Tinn-R, the ultimate editor for \RR{}
users on Windows.\\

The functionality of Tinn-R goes far beyond that of a simple text editor;
it allows you to define projects, highlight important syntax elements,
and send \RR{} code to the console. Using Tinn-R allows you to be much more
productive when working with \RR{}.\\

The book is divided into four parts: Overview, Basics, Working With and 
Menu Description. It not only provides a very readable introduction to
Tinn-R, but also serves as a valuable reference.\\

I hope you enjoy it!

\begin{flushright}\noindent
  Diethelm W�rtz \\
  Z�rich \\
  10 November 2010
\end{flushright}


\newpage
\section*{Preface from the Editus - 2. ed.}
\addcontentsline{toc}{section}{Editus (2. ed.)}

It is a pleasure to introduce the first eBook in the new \textbf{Editus eBook Series}.
This is a series of eBooks addressed to students and researchers of arts and sciences in general.\\

Tinn-R is one of the most successful graphical interfaces for \RR{},
and it has been programmed in its entirety at Universidade Estadual de Santa Cruz
by Jos� Cl�udio Faria which is a professor of this institution.\\

This 2. ed. has been completely reviewed and updated according to the latest version of Tinn-R project.\\

The book is aimed to \RR{} users and programmers of any level, from the beginner to the most advanced.
In order to help the beginner a new chapter called \textit{Some Secrets for an Efficient Use} has been added.\\

I hope you enjoy it!

\begin{flushright}\noindent
  Rita Virginia Argollo \\
  Ilh�us - Bahia - Brasil \\
  08 January 2014
\end{flushright}

