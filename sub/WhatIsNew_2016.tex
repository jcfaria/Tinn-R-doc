
\hypertarget{2016}{}
\section{Versions released in 2016 (01)}
\index{What is New?!2016}

\subsection*{5.1.2.0 (oct/15/2016)}
\begin{itemize}
  \item Bug(s) fixed:
    \begin{itemize}
      \item Two major bugs introduced in the \texttt{pre-release version 4.2.6.0} have been fixed.
      \item A bug related to \texttt{Help}, \texttt{Example} and \texttt{Open example}
        when all files were closed was fixed.
      \item A bug related to \texttt{Rterm interface (IO)} and the package \texttt{debug} was fixed.
        Thanks to \texttt{Ivan B. Allaman} for pointing it out.
      \item A bug related to the \texttt{custom color storage} was fixed.
      \item A bug related to the \texttt{Options/Syntax (highlighter)/Default (to new files)} was fixed.
      \item A bug related to the \texttt{Options/Syntax (highlighter)} was fixed.
        Thanks to \texttt{Manuela Huso} for pointing it out.
      \item Some issues about the \texttt{completion} followed by \texttt{F3}
        to find the next occurrence of \texttt{|} were corrected.
      \item Issues about the installation of TinnRcom package (and its dependences) were corrected.
    \end{itemize}
  \item A lot of \texttt{PRE-RELEASE} versions of the project have been released, not restrict to testers. Thanks for testing and suggestions.
  \item The installation of TinnRcom package (and its dependences) was improved. From now, the sources (\texttt{.tar.gz}) and
    the binaries (\texttt{.zip}) of the packages \texttt{formatR}, \texttt{svMisc}, \texttt{svSocket} and \texttt{TinnRcom}
    will be released within the setup of Tinn-R. After the installation, it will be placed at \texttt{packages} folder.
    So all will be installed from the local \texttt{.zip}.
  \item The package \texttt{TinnRcom} was upgraded to the version \texttt{1.0.20}.
  \item The menu \texttt{options} has a new option: \texttt{R echo (on/off)}. This grants the user the choice to echo (or not)
    some options of send file, selection, clipboard, block marked, contiguous lines a line to end of page.
    A related button was add to R task bar. This made simpler and user friendly these options.
    The default shortcut to toggle this option is \texttt{ALT + E}
  \item The procedure to \texttt{open a remote file from an URL} was improved and also covers  \texttt{https} protocol.
  \item If the user choice in \texttt{Options/Application/R/Patch (R)} is
    \textbf{No} to \texttt{Use latest installed version (always)} option, at startup,
    Tinn-R will search in all letters of the system drives for the fully informed path of R.
    It is very useful in the portable flavours due to letters changes in different computers.
  \item The visibility of all \texttt{Page Control} and \texttt{Tab Sheets} caption were improved.
  \item R will always start with two options:
    \begin{itemize}
      \item \texttt{options(pkgType='binary')}
      \item \texttt{options(install.packages.check.source='no')}
    \end{itemize}
  \item If the user do not have a personal library to manage the packages,
    from this version on, Tinn-R will create a folder named \texttt{x.y}, related to major and minor R version in the \texttt{C:/Users/User/Documents/R/win-library/}.
    This library will be used as default to package manager. The user can change the default library at
    \texttt{Options/Application/R/Packages (R)} in the \texttt{Library trees (.libPaths()}.
  \item The default files \texttt{Shortcuts.xml} and \texttt{Mirrors.xml} were updated.
  \item The default shortcuts related to focus on \texttt{Rterm interface} (Editor, IO and LOG) were changed
    due conflict with Windows 10.
  \item When using \texttt{Rterm} (except with the \texttt{Send File}option) all other options related with more than one line will be added do the \texttt{R history}.
  \item The source code related to identification of \texttt{library trees .libPaths()} was improved.
  \item The \texttt{R explorer} interface related to identification of environments from objects was improved.
  \item The resources related to \texttt{update mirrors} in runtime were improved.
  \item The resources related to \texttt{knitr} and \texttt{Sweave} were improved and new options are in \texttt{Options/Application/R/Packages/Knitr}.
  \item To all \texttt{knit} procedures it will be added the argument \texttt{quiet=TRUE}.
    So, if you want more control, or to big documents, it is suggested (for while) to use the \texttt{knit}
    with \texttt{Rgui.exe} instead of \texttt{Rterm.exe}.
  \item The menu \texttt{Insert} has a new option: \texttt{R (assignment)}. This option make flexible
    to insert \texttt{<-} and \texttt{->} assignment. As it is associated to shortcuts, the user can customize both.
  \item The navigator of all database was replaced.
  \item The \texttt{IO} prompt of the \texttt{Rterm} interface is more user friendly. For this
    set \texttt{Options/Application/Editor/Advanced/Scroll pas end of line} option.
  \item It was add a new item in the main menu: \texttt{Tools/Processing/Viewer/Open current file (generic)}.
  \item The menu \texttt{Web} as heavy reworked and has new options.
  \item The windows \texttt{Options/Application} received improvements.
  \item A new resource allowing the user to \texttt{open/learning/edit} the example script of any R objet,
    from the \texttt{editor}, \texttt{Rterm/IO}, \texttt{Rterm/LOG} or window \texttt{Tools/R/explorer} was added.
  \item The \texttt{tip} and \texttt{data completion} were deeply improved.
    \texttt{CTRL + SPACE} is the single trigger for both and the distinction is made in the context of the call.
    Additionally this feature now find the object, the package and the pattern in complex contexts multiline.
  \item The shortcuts \texttt{CTRL + TAB} and \texttt{SHIFT + CTRL + TAB}, not user configurable, can now be also used in all interfaces
    \texttt{Editor, Rterm and Help} to sequentially switch the pages.
  \item This version is full compatible with \texttt{Microsoft R Open - MRO}.
  \item The folder \texttt{utils} was deeply restructured and it was added an instruction \texttt{unlockBinding("last.warning", baseenv())}
    to the new file \texttt{info.R} to workaround a small bug among \texttt{svSocket} and \texttt{MRO}.
    Thanks to \texttt{Marc Laurencelle} to pointing it out.
  \item The identifier \texttt{Note} brought back to the highlighter of R family, and due to this utility also added to \texttt{Text} highlighter. Thanks to \texttt{Ari} for pointing it out.
  \item The \texttt{R family, All and Text highlighter} received improvements related to \texttt{strings} identification.
    The intention is that they do not identify the shortened forms (he's, you'd, you've, etc.) as string.
  \item New options related to \texttt{Rterm: IO and LOG} highlighters.
  \item The usability of \texttt{Rterm} interface is now more robust, stable and user friendly.
  \item An new resource \texttt{Check for update} was added to the main menu \texttt{Help}.
  \item The development team would like to publicly thank \texttt{Marc Laurencelle} for the contributions and suggestions to Tinn-R project.
  \item Tinn-R Team has two new members: \texttt{Philiphe A. Kramer} and \texttt{Swami de P. Lima}, welcome!
\end{itemize}