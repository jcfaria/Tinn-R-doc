
\hypertarget{2016}{}
\section{Versions released in 2016 (01)}
\index{What is New?!2016}

\subsection*{5.1.2.0 (oct/15/2016)}
\begin{itemize}
  \item Bug(s) fixed:
    \begin{itemize}
      \item Two major bugs introduced in the \texttt{pre-release version 4.2.6.0} have been fixed.
      \item A bug related to \texttt{Help}, \texttt{Example} and \texttt{Open example}
        when all files were closed was fixed.
      \item A bug related to \texttt{Rterm interface (IO)} and the package \texttt{debug} was fixed.
        Thanks to \texttt{Ivan B. Allaman} for pointing it out.
      \item A bug related to the \texttt{custom color storage} was fixed.
      \item A bug related to the \texttt{Options/Syntax (highlighter)/Default (to new files)} was fixed.
      \item A bug related to the \texttt{Options/Syntax (highlighter)} was fixed.
        Thanks to \texttt{Manuela Huso} for pointing it out.
      \item Some issues about the \texttt{completion} followed by \texttt{F3}
        to find the next occurrence of \texttt{|} were corrected.
      \item Issues about the installation of TinnRcom package (and its dependencies) were corrected.
    \end{itemize}
  \item A lot of \texttt{PRE-RELEASE} versions of the project have been released, not restricted to testers. Thanks for testing and making suggestions.
  \item The installation of TinnRcom package (and its dependencies) was improved. From now on, the sources (\texttt{.tar.gz}) and
    the binaries (\texttt{.zip}) of the packages \texttt{formatR}, \texttt{svMisc}, \texttt{svSocket} and \texttt{TinnRcom}
    will be released within the Tinn-R setup. After the installation, TinnRcom package will be placed at \texttt{packages} folder.
    Thus all of them will be installed from the local \texttt{.zip}.
  \item The \texttt{TinnRcom} package was upgraded to version \texttt{1.0.20}.
  \item The menu \texttt{options} has a new option: \texttt{R echo (on/off)}. This grants the user the choice to echo (or not)
    some options of send file, selection, clipboard, block marked, contiguous lines a line to end of page.
    A related button was added to R task bar. This made these options simpler and user-friendly.
    The default shortcut to toggle this option is \texttt{ALT + E}
  \item The procedure to \texttt{open a remote file from an URL} was improved and covers \texttt{https} protocol.
  \item If the user's choice in \texttt{Options/Application/R/Patch (R)} is
    \textbf{No} to \texttt{Use the latest installed version (always)} option, at startup,
    Tinn-R will search all letters of the system drives for the fully informed path of R.
    This is very useful in the portable flavours due to letters changes in different computers.
  \item The visibility of all \texttt{Page Control} and \texttt{Tab Sheets} caption was improved.
  \item R will always start with two options:
    \begin{itemize}
      \item \texttt{options(pkgType='binary')}
      \item \texttt{options(install.packages.check.source='no')}
    \end{itemize}
  \item If the user do not have a personal library to manage the packages,
    from this version on, Tinn-R will create a folder named \texttt{x.y}, related to major and minor R version in the \texttt{C:/Users/User/Documents/R/win-library/}.
    This library will be used as default to package manager. The user can change the default library at
    \texttt{Options/Application/R/Packages (R)} in the \texttt{Library trees (.libPaths()}.
  \item The default files \texttt{Shortcuts.xml} and \texttt{Mirrors.xml} were updated.
  \item The default shortcuts related to focus on \texttt{Rterm interface} (Editor, IO and LOG) were changed
    due to conflict with Windows 10.
  \item When using \texttt{Rterm} (except for the \texttt{Send File}option), all options related to more than one line will be added to the \texttt{R history}.
  \item The source code related to identification of \texttt{library trees .libPaths()} was improved.
  \item The \texttt{R explorer} interface related to the identification of environments from objects was improved.
  \item The resources related to \texttt{update mirrors} in runtime were improved.
  \item The resources related to \texttt{knitr} and \texttt{Sweave} were improved and new options are in \texttt{Options/Application/R/Packages/Knitr}.
  \item The argument \texttt{quiet=TRUE} it will be added to all \texttt{knit} procedures.
    So, if you want more control, or for big documents, (for while) the use of the \texttt{knit}
    with \texttt{Rgui.exe} instead of \texttt{Rterm.exe} is suggested.
  \item The menu \texttt{Insert} has a new option: \texttt{R (assignment)}. This option makes flexible
    to insert \texttt{<-} and \texttt{->} assignment. As it is associated with shortcuts, the user can customize both assignments.
  \item The navigator of all database was replaced.
  \item The \texttt{IO} prompt of the \texttt{Rterm} interface is more user friendly. For this
    set \texttt{Options/Application/Editor/Advanced/Scroll pas end of line} option.
  \item A new item was add to the main menu: \texttt{Tools/Processing/Viewer/Open current file (generic)}.
  \item The menu \texttt{Web} was heavily reworked and has new options.
  \item The windows \texttt{Options/Application} were improved.
  \item A new resource allowing the user to \texttt{open/learn/edit} the example script of any R objet,
    from the \texttt{editor}, \texttt{Rterm/IO}, \texttt{Rterm/LOG} or window \texttt{Tools/R/explorer} was added.
  \item The \texttt{tip} and \texttt{data completion} were deeply improved.
    \texttt{CTRL + SPACE} is the single trigger for both and the distinction is made in the context of the call.
    Additionally this feature now find the object, the package and the pattern in complex multiline contexts.
  \item The shortcuts \texttt{CTRL + TAB} and \texttt{SHIFT + CTRL + TAB}, not user configurable, can now be used in all interfaces
    \texttt{Editor, Rterm and Help} to sequentially switch the pages.
  \item This version is full compatible with \texttt{Microsoft R Open - MRO}.
  \item The folder \texttt{utils} was deeply restructured an instruction \texttt{unlockBinding("last.warning", baseenv())}
    was added to the new file \texttt{info.R} to workaround a small bug between \texttt{svSocket} and \texttt{MRO}.
    Thanks to \texttt{Marc Laurencelle} for pointing it out.
  \item The identifier \texttt{Note} was brought back to the highlighter of R family, and due to this utility, it was also added to \texttt{Text} highlighter.
    Thanks to \texttt{Ari} for pointing it out.
  \item Improvements related to identification of \texttt{strings} were made to \texttt{R family}, \texttt{All} and \texttt{Text} highlighters.
    So, they do not identify the shortened forms (he's, you'd, you've, etc.) as string.
  \item New options related to \texttt{Rterm: IO and LOG} highlighters.
  \item The usability of \texttt{Rterm} interface is now more robust, stable and user friendly.
  \item A new resource \texttt{Check for update} was added to the main menu \texttt{Help}.
  \item The development team would like to publicly thank \texttt{Marc Laurencelle} for the contributions and suggestions to Tinn-R project.
  \item Tinn-R Team has two new members: \texttt{Philiphe A. Kramer} and \texttt{Swami de P. Lima}, welcome!
\end{itemize}